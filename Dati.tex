%----------------------------------------------------------------------------------------
%   USEFUL COMMANDS
%----------------------------------------------------------------------------------------

\newcommand{\dipartimento}{Dipartimento di Matematica ``Tullio Levi-Civita''}

%----------------------------------------------------------------------------------------
% 	USER DATA
%----------------------------------------------------------------------------------------

% Data di approvazione del piano da parte del tutor interno; nel formato GG Mese AAAA
% compilare inserendo al posto di GG 2 cifre per il giorno, e al posto di 
% AAAA 4 cifre per l'anno
\newcommand{\dataApprovazione}{Data}

% Dati dello Studente
\newcommand{\nomeStudente}{Riccardo}
\newcommand{\cognomeStudente}{Fabbian}
\newcommand{\matricolaStudente}{2009110}
\newcommand{\emailStudente}{riccardo.fabbian@studenti.unipd.it}
\newcommand{\telStudente}{+ 39 339 445 6099}

% Dati del Tutor Aziendale
\newcommand{\nomeTutorAziendale}{Ombretta}
\newcommand{\cognomeTutorAziendale}{Gaggi}
\newcommand{\emailTutorAziendale}{gaggi@math.unipd.it}
\newcommand{\telTutorAziendale}{+ 39 000 00 00 000}
\newcommand{\ruoloTutorAziendale}{}

% Dati dell'Azienda
\newcommand{\ragioneSocAzienda}{Azienda S.p.A}
\newcommand{\indirizzoAzienda}{Via Roma 1, Roma (RM)}
\newcommand{\sitoAzienda}{http://example.com/}
\newcommand{\emailAzienda}{mail@mail.it}
\newcommand{\partitaIVAAzienda}{P.IVA 12345678999}

% Dati del Tutor Interno (Docente)
\newcommand{\titoloTutorInterno}{Prof.}
\newcommand{\nomeTutorInterno}{NomeDocente}
\newcommand{\cognomeTutorInterno}{CognomeDocente}

\newcommand{\prospettoSettimanale}{
     % Personalizzare indicando in lista, i vari task settimana per settimana
     % sostituire a XX il totale ore della settimana
    \begin{itemize}
        \item \textbf{Prima Settimana (XX ore)}
        \begin{itemize}
            \item Incontro con persone coinvolte nel progetto per discutere i requisiti e le richieste
            relativamente al plugin da sviluppare;
            \item Studio e/o ripasso delle tecnologie coinvolte nello sviluppo del plugin;
            \item Analisi approfondita delle normative sull'accessibilità e delle best practices;
        \end{itemize}
        \item \textbf{Seconda Settimana - Sottotitolo (XX ore)} 
        \begin{itemize}
            \item Studio e progettazione dell'architettura del plugin;
            \item Inizio dello sviluppo del core del plugin per l'identificazione delle immagini;
        \end{itemize}
        \item \textbf{Terza Settimana - Sottotitolo (XX ore)} 
        \begin{itemize}
            \item ;
        \end{itemize}
        \item \textbf{Quarta Settimana - Sottotitolo (XX ore)} 
        \begin{itemize}
            \item ;
        \end{itemize}
        \item \textbf{Quinta Settimana - Sottotitolo (XX ore)} 
        \begin{itemize}
            \item ;
        \end{itemize}
        \item \textbf{Sesta Settimana - Sottotitolo (XX ore)} 
        \begin{itemize}
            \item ;
        \end{itemize}
        \item \textbf{Settima Settimana - Sottotitolo (XX ore)} 
        \begin{itemize}
            \item ;
        \end{itemize}
        \item \textbf{Ottava Settimana - Conclusione (XX ore)} 
        \begin{itemize}
            \item ;
        \end{itemize}
    \end{itemize}
}

% Indicare il totale complessivo (deve essere compreso tra le 300 e le 320 ore)
\newcommand{\totaleOre}{}

\newcommand{\obiettiviObbligatori}{
	 \item \underline{\textit{O01}}: Approfondimento delle tematiche relative all'accessibilità web;
	 \item \underline{\textit{O02}}: Abilità nell'autogestire il progetto e nell'adempire agli obiettivi stabiliti;
	 \item \underline{\textit{O03}}: Sviluppo di un prodotto finale di qualità utilizzabile per l'analisi statica dell'accessibilità di un sito web;
	 
}

\newcommand{\obiettiviDesiderabili}{
	 \item \underline{\textit{D01}}: ...;
}

\newcommand{\obiettiviFacoltativi}{
	 \item \underline{\textit{F01}}: Pubblicazione del plugin all'interno del Chrome Web Store;
}